\documentclass[
    12pt,               % general font size
    aspectratio=1610,   % aspect ratio of presentation
    compress,           % compress PDF?
    handout,            % if active: squash subslides into single frames
    ccby                % display cc-by icon on front page
    ]{beamer}
\usepackage[german]{babel}
\usepackage{tikz}
\usetheme{Goethe}

\makeatletter
\def\@affiliation{}
\def\@contactaddress{}
\def\@contactmail{}
\def\@contactweb{}
\makeatother

\usepackage[square]{natbib}

\author{Xinran Wang}

\title{Seminarpräsentation: Polynomial Time Data Reduction for Dominating Set}

\date{Frankfurt am Main, \today}

% set affiliation and contact in front matter
\affiliation{Goethe University Frankfurt}
\contactmail{s1922410@stud.uni-frankfurt.de}
  
\begin{document}

% --- Titelfolie ---
\begin{frame}  % define frame type title for title page background
    \maketitle
\end{frame}

% --- Gliederungsfolie ---
\begin{frame}
    \frametitle{Gliederung}
    \tableofcontents
\end{frame}

\tikzset{
    knoten/.style={
        circle,
        fill=black,
        text=white,
        inner sep=2pt,
        minimum size=22pt,
        font=\large
    }
}

%---------------------------------------------------
\section{Motivation: Das Dominating-Set-Problem}
%---------------------------------------------------
\begin{frame}{Motivation \& Problemstellung}
    \begin{block}{Das Dominating-Set-Problem}
        Gegeben sei ein ungerichteter Graph \( G \). Ein \( k \)-Dominating-Set ist eine Menge von \( k \) Knoten, sodass jeder nicht enthaltene Knoten mindestens einen Nachbarn in dieser Menge hat. Die Dominationszahl \( \gamma(G) \) ist die kleinste Größe eines dominierenden Sets. Das Dominating-Set-Problem fragt, ob \( \gamma(G) \leq k \) gilt.

    \end{block}

    \begin{columns}[T]
        \begin{column}{0.5\textwidth}
            \begin{alertblock}{Eine große Herausforderung}
                \begin{itemize}
                    \item Das Problem ist NP-schwer und W[2]-vollständig. 
                    \item Das bedeutet: Ein schneller Algorithmus für allgemeine Graphen ist höchst unwahrscheinlich. \cite{W-hierarchy}
                \end{itemize}
            \end{alertblock}
        \end{column}
        \begin{column}{0.5\textwidth}
            \begin{exampleblock}{Praktische Relevanz}
                \begin{itemize}
                    \item In mobilen Ad-hoc-Netzwerken (MANETs) wird eine kleine dominierende Menge als „virtuelles Rückgrat" für die Kommunikation genutzt. 
                \end{itemize}
            \end{exampleblock}
        \end{column}
    \end{columns}
\end{frame}

\begin{frame}{Die Lösungsstrategie: Datenreduktion}
    \begin{block}{Die zentrale Idee}
        Anstatt das schwere Problem direkt zu lösen, wenden wir eine Vorverarbeitung (Preprocessing) an. 
    \end{block}
    \begin{itemize}
        \item \textbf{Ziel:} Die Problemgröße durch einfache Reduktionsregeln gezielt verkleinern. 
        \item \textbf{Bedingung:} Die Dominationszahl $\gamma(G)$ des Graphen darf dabei nicht verändert werden. 
        \item \textbf{Grundlage:} Die Arbeit von Alber, Fellows und Niedermeier, die zwei intuitive Regeln vorschlägt. \cite{PolynomialTimeDataReduction}
    \end{itemize}
\end{frame}

%---------------------------------------------------
\section{Grundlagen \& Definitionen}
%---------------------------------------------------

\begin{frame}{Graphentheoretische Begriffe}
\begin{definition}[Nachbarschaft und Grad]
Sei $G = (V, E)$ ein Graph.
Zwei Knoten $i, j \in V$ heißen \textbf{adjazent} oder \textbf{benachbart}, wenn eine Kante $\{i, j\} \in E$ existiert.
Die \textbf{offene Nachbarschaft} eines Knotens $i \in V$ ist die Menge all seiner Nachbarknoten und wird mit $N(i)$ bezeichnet:
    \[
        N(i) = \{j \in V \mid \{i, j\} \in E\}
    \]
Die \textbf{geschlossene Nachbarschaft} eines Knotens $i \in V$ ist die Menge, die $i$ selbst und alle seine Nachbarn enthält und wird mit $N[i]$ bezeichnet:
    \[
        N[i] = N(i) \cup \{i\}
    \]
Der \textbf{Grad} eines Knotens $i$, bezeichnet mit $\deg(i)$, ist die Anzahl seiner Nachbarn, also $\deg(i) = |N(i)|$.

\end{definition}
\end{frame}

\begin{frame}{Graphentheoretische Begriffe}
\begin{definition}[Paar-Nachbarschaft]
Sei $G=(V,E)$ ein Graph und seien $i, j \in V$ zwei verschiedene Knoten.
Die Menge der \textit{kombinierten Nachbarn} von $i$ und $j$ ist die Menge aller Knoten, die zu mindestens einem der beiden Knoten $i$ oder $j$ adjazent sind. Gemäß der hier verwendeten Quelle wird sie mit $N(i,j)$ bezeichnet:
\[
    N(i,j) = N(i) \cup N(j)
\]
\end{definition}
\end{frame}

%---------------------------------------------------
\section{Reduktionsregel 1 (Fokus: Einzelner Knoten)}
%---------------------------------------------------
\begin{frame}{Reduktionsregel 1}
Wir partitionieren die Nachbarn $N(v)$ in drei disjunkte Mengen: 

\begin{align*}
    N_{1}(v) &:= \{u \in N(v) : N(u) \setminus N[v] \ne \emptyset\}, \\
    N_{2}(v) &:= \{u \in N(v) \setminus N_{1}(v) : N(u) \cap N_{1}(v) \ne \emptyset\}, \\
    N_{3}(v) &:= N(v) \setminus (N_{1}(v) \cup N_{2}(v)).
\end{align*}
\begin{center}
\begin{tikzpicture}[scale=0.7,
    % Design von Knotenstyle
    v_style/.style={circle, draw, fill=black, text=white, inner sep=2pt, minimum size=4mm},
    n1_style/.style={circle, draw, fill=red, inner sep=3pt, minimum size=3mm}, % N1 ist rot
    n2_style/.style={circle, draw, fill=blue, inner sep=3pt, minimum size=3mm}, % N2 ist blau
    n3_style/.style={circle, draw, fill=white, inner sep=3pt, minimum size=3mm}, % N3 bleibt weiss
    legend_text/.style={anchor=west} 
]
    % Zentrumknoten v
    \node[v_style] (v) at (0,0) {$v$};

    % N1 Knoten: rot
    \node[n1_style] (n1a) at (0:2.5cm) {};
    \node[n1_style] (n1b) at (135:2.5cm) {};
    \node[n1_style] (n1c) at (180:2.5cm) {};

    % N2 Knoten: blau
    \node[n2_style] (n2a) at (45:2.5cm) {};
    \node[n2_style] (n2b) at (-45:2.5cm) {};
    \node[n2_style] (n2c) at (-110:2.5cm) {};
    \node[n2_style] (n2d) at (-135:2.5cm) {};

    % N3 Knoten: weiss
    \node[n3_style] (n3a) at (85:2.5cm) {};
    \node[n3_style] (n3b) at (-85:2.5cm) {};

    % Alle Knoten mit Zentrumknoten verbinden
    \foreach \point in {n1a, n1b, n1c, n2a, n2b, n2c, n2d, n3a, n3b}
    {
        \draw (v) -- (\point);
    }
    % Zwischenkanten
    \draw (n1a) -- (n2a);
    \draw (n1a) -- (n2b);
    \draw (n2a) -- (n3a);
    \draw (n1b) -- (n1c);
    \draw (n1c) -- (n2d);
    \draw (n1c) -- (n2c);
    \draw (n2d) -- (n2c);
    \draw (n2c) -- (n3b);
    \draw (n1a) -- ++(0:1cm);
    \draw (n1a) -- ++(30:1cm);
    \draw (n1b) -- ++(135:1cm);
    \draw (n1c) -- ++(170:1cm);
    \draw (n1c) -- ++(190:1cm);

    \node[n1_style] (leg1) at (-6, 1) {};
    \node[legend_text] at ([xshift=0.5cm]leg1.east) {$N_1(v)$};
    
    \node[n2_style] (leg2) at (-6, 0) {};
    \node[legend_text] at ([xshift=0.5cm]leg2.east) {$N_2(v)$};
    
    \node[n3_style] (leg3) at (-6, -1) {};
    \node[legend_text] at ([xshift=0.5cm]leg3.east) {$N_3(v)$};

\end{tikzpicture}
\end{center}
\end{frame}

\begin{frame}{Reduktionsregel 1}
    \begin{block}{Bedingung}
        Die Reduktionsregel wird auf einen Knoten $v \in V$ angewendet, sofern dessen Partition $N_3(v)$ nicht leer ist.
    \end{block}
    
    \textbf{Aktion: Transformation zu $G'$}
    \begin{itemize}
        \item Entferne alle Knoten aus $N_2(v)$ und $N_3(v)$. 
        \item Füge einen neuen "Gadget-Knoten" $v'$ und die Kante $\{v,v'\}$ hinzu. 
    \end{itemize}

\begin{center}
\begin{tikzpicture}[scale=0.7,
    v_style/.style={circle, draw, fill=black, text=white, inner sep=2pt, minimum size=4mm},
    n1_style/.style={circle, draw, fill=red, inner sep=3pt, minimum size=3mm}, % N1 ist rot
]
    \node[v_style] (v) at (0,0) {$v$};
    \node[v_style] (v') at (2,2) {$v'$};
    % N1 Knoten: rot
    \node[n1_style] (n1a) at (0:2.5cm) {};
    \node[n1_style] (n1b) at (135:2.5cm) {};
    \node[n1_style] (n1c) at (180:2.5cm) {};

    \foreach \point in {n1a, n1b, n1c, v'}
    {
        \draw (v) -- (\point);
    }
    \draw (n1b) -- (n1c);
    \draw (n1a) -- ++(0:1cm);
    \draw (n1a) -- ++(30:1cm);
    \draw (n1b) -- ++(135:1cm);
    \draw (n1c) -- ++(170:1cm);
    \draw (n1c) -- ++(190:1cm);
\end{tikzpicture}
\end{center}
\end{frame}

\begin{frame}{Korrektheit von Regel 1}
\begin{lemma}
Gegeben sei ein Graph \( G(V, E) \), und Graph  \( G'(V', E') \) der Graph, der durch Anwendung der Reduktionsregel 1 entsteht, dann gilt: \( \gamma(G) = \gamma(G') \).
\end{lemma} 
\begin{itemize}
    \item \( N_3(v) \) können nur von \( v \) oder von Knoten aus \( N_2(v) \cup N_3(v) \) dominiert werden
    \item Für jedes \( w \in N_2(v) \cup N_3(v) \) gilt \( N(w) \subseteq N(v) \)
    \item Die Nachbarn der Knoten aus \( N_2(v) \cup N_3(v) \) ebenfalls Nachbarn von \( v \) sind 
    \item Alle Knoten, die von $N_2(v) \cup N_3(v)$ dominiert werden, auch automatisch von $v$ dominiert
    \item Mit $v$ eine genau so gut oder bessere Menge konstruieren. $v'$ erzingt, dass \( v \) in eine optimale dominante Menge im reduzierten Graphen aufgenommen wird
    \item Das Löschen von $N_2(v) \cup N_3(v)$ ist eine korrekte Reduktion
    \item \(=>\) \( \gamma(G') = \gamma(G) \)
\end{itemize}
\end{frame}

\begin{frame}[shrink]{Laufzeit von Regel 1}
    \begin{block}{Planare Graphen}
        Gesamtlaufzeit: $O(n)$ 
        \begin{itemize}
            \item Die Analyse der Nachbarschaftsbeziehung der geschlossenen Nachbarschaft $N[v] = N(v) \cup \{v\}$ \(=>\) Die relevante Struktur: der von $N[v]$ induzierte Teilgraph, $G[N[v]]$
            \item Knotenanzahl von $G[N[v]]$ ist $k = |N[v]| = \deg(v) + 1$ \(=>\) $O(\deg(v))$
            \item $G$ planar \(=>\) $G[N[v]]$ auch planar
            \item Euler-Satz: ein einfacher planarer Graph mit $k \ge 3$ Knoten besitzt höchstens $3k-6$ Kanten
            \item Die Anzahl der Kanten in $G[N[v]]$: durch $3(\deg(v)+1) - 6 = 3\deg(v)-3$ beschränkt \(=>\) $O(\deg(v))$
            \item Die Bestimmung von $N_1(v), N_2(v), N_3(v)$ durch die Untersuchung aller Knoten und Kanten in $G[N[v]]$
            \item Die Laufzeit für Verarbeitung eines einzelnen Knotens $v$ ebenfalls $O(\deg(v))$
            \item Die Gesamtelaufzeit: Die Summe der Kosten über alle Knotens des $G[N[v]]$:
            \[
            \sum_{v \in V} O(\deg(v)) = O\left(\sum_{v \in V} \deg(v)\right)
            \]
            \item Die Gesamtlaufzeit in $O(n)$ liegt.           
        \end{itemize}
    \end{block}
\end{frame}  

\begin{frame}[shrink]{Laufzeit von Regel 1}
    \begin{block}{Allgemeine Graphen}
        Gesamtlaufzeit: $O(n^3)$ 
        \begin{itemize}
            \item Die Anzahl der Knoten: $k = \deg(v)+1$
            \item Die Anzahl der Kanten (Im schlimmsten Fall: ein Clique): $O(k^2) = O(\deg(v)^2)$
            \item Die Analyse der Nachbarschaftsbeziehungen für einen Knoten $v$ erfordert daher eine Laufzeit von $O(\deg(v)^2)$
            \item Die Gesamtlaufzeit aller Knoten erfolgt
            \[
            \sum_{v \in V} \mathcal{O}((\mathrm{deg}(v))^2).
            \]
            \item Für jeden Knoten \( v \in V \) gilt: \( \mathrm{deg}(v) < n \), also
            \[
            (\mathrm{deg}(v))^2 < n^2.
            \]
            \item Die obere Schranke für einen einzelnen Knoten
            \[
            (\mathrm{deg}(v))^2 = \mathcal{O}(n^2).
            \]
            \item In der Gesamtsumme jeden Summanden durch diese obere Schranke ersetzen
            \[
            \sum_{v \in V} \mathcal{O}((\mathrm{deg}(v))^2) \leq \sum_{v \in V} \mathcal{O}(n^2) = \mathcal{O}(n^3).
            \]
            \item Die Gesamtlaufzeit für allgemeine Graphen durch $O(n^3)$ beschränkt
        \end{itemize}
    \end{block}
\end{frame}


%---------------------------------------------------
\section{Reduktionsregel 2 (Fokus: Knotenpaar)}
%---------------------------------------------------
\begin{frame}{Motivation für Regel 2}
    \begin{columns}
        \begin{column}{0.5\textwidth}
            \begin{block}{Ein Problem für Regel 1}
                \begin{itemize}
                    \item In diesem Graphen hat kein einziger Knoten eine nicht-leere $N_3$-Menge. 
                    \item Zum Beispiel sind die Nachbarn von $v$ (also $x$ und $y$) über $w$ mit dem Rest des Graphen verbunden.
                    \item \textbf{Regel 1 ist nicht anwendbar.} 
                \end{itemize}
            \end{block}
            \begin{alertblock}{Die Lösung}
                Wir müssen die Struktur von \textbf{Knotenpaaren}, hier $(v,w)$, gleichzeitig betrachten. 
            \end{alertblock}
        \end{column}
        \begin{column}{0.5\textwidth}
             \begin{tikzpicture}[scale=0.7, transform shape]
    
                % Die vier Knoten an den Ecken einer Raute platzieren
                % (Knotenname) at (Koordinaten) {Label}
                \node[knoten] (v) at (-2.5, 0) {v};
                \node[knoten] (w) at ( 2.5, 0) {w};
                \node[knoten] (x) at ( 0, 2) {x};
                \node[knoten] (y) at ( 0,-2) {y};
            
                % Die Kanten zwischen den Knoten zeichnen
                \draw[thick] (v) -- (x);
                \draw[thick] (v) -- (y);
                \draw[thick] (w) -- (x);
                \draw[thick] (w) -- (y);
            
            \end{tikzpicture}
        \end{column}
    \end{columns}
\end{frame}

\begin{frame}{Reduktionsregel 2}
    \begin{itemize}
        \item Für ein Knotenpaar $(v,w)$ analysieren wir die kombinierte Nachbarschaft $N(v,w) := N(v) \cup N(w)$. 
        \item Diese wird analog zu Regel 1 partitioniert in $N_1(v,w), N_2(v,w)$ und $N_3(v,w)$. 
        \begin{align*}
            N_{1}(v,w) &:= \{u \in N(v,w) : N(u) \setminus N[v,w] \ne \emptyset\}, \\
            N_{2}(v,w) &:= \{u \in N(v,w) \setminus N_{1}(v,w) : N(u) \cap N_{1}(v,w) \ne \emptyset\}, \\
            N_{3}(v,w) &:= N(v,w) \setminus (N_{1}(v,w) \cup N_{2}(v,w)).
        \end{align*}
    \end{itemize}
\end{frame}    
    
\begin{frame}{Reduktionsregl 2}
Regel 2 wird angewendet, wenn $N_3(v,w)$ nicht-leer ist und keine einfache Einzelpunktlösung existiert. \\
Folgendes Beispiel visualisiert die Partitionierung:
    \begin{center}
        \includegraphics[width=0.75\textwidth]{seminar_paar_Knoten_Beispiel.png} 
    \end{center}
\end{frame}

\begin{frame}{Reduktionsregel 2: Fall 1 und Fall 2}
    \begin{itemize}
        \item[Fall 1] Wenn \(N_{3}(v,w)\) von einem einzelnen Knoten aus \(\{v, w\}\) dominiert werden kann:
        \begin{enumerate}
            \item[(1.1)] Wenn \(N_{3}(v,w) \subseteq N(v)\) und zugleich \(N_{3}(v,w) \subseteq N(w)\) gilt:
                \begin{itemize}
                    \item Entferne \(N_{3}(v,w)\) und \(N_{2}(v,w) \cap N(v) \cap N(w)\) aus G.
                    \item Füge zwei neue Knoten \(z, z'\) und die Kanten \(\{v,z\}, \{w,z\}, \{v,z'\}, \{w,z'\}\) zu G hinzu.
                \end{itemize}

            \item[(1.2)] Wenn \(N_{3}(v,w) \subseteq N(v)\), aber nicht \(N_{3}(v,w) \subseteq N(w)\) gilt:
                 \begin{itemize}
                    \item Entferne \(N_{3}(v,w)\) und \(N_{2}(v,w) \cap N(v)\) aus G.
                    \item Füge einen neuen Knoten \(v'\) und die Kante \(\{v,v'\}\) zu G hinzu.
                \end{itemize}
            
            \item[(1.3)] Wenn \(N_{3}(v,w) \subseteq N(w)\), aber nicht \(N_{3}(v,w) \subseteq N(v)\) gilt:
                 \begin{itemize}
                    \item Entferne \(N_{3}(v,w)\) und \(N_{2}(v,w) \cap N(w)\) aus G.
                    \item Füge einen neuen Knoten \(w'\) und die Kante \(\{w,w'\}\) zu G hinzu.        
                \end{itemize}
        \end{enumerate}

        \item[Fall 2] Wenn \(N_{3}(v,w)\) nicht von einem einzelnen Knoten aus \(\{v, w\}\) dominiert werden kann:
        \begin{itemize}
            \item Entferne \(N_{3}(v,w)\) und \(N_{2}(v,w)\) aus G.
            \item Füge zwei neue Knoten \(v', w'\) und die Kanten \(\{v,v'\}, \{w,w'\}\) zu G hinzu.
        \end{itemize}    
    \end{itemize}
\end{frame}
    
\begin{frame}{Reduktionsregel 2: Korrektheit beweisen}   
Die zentrale Aufgaben: \(N_3(v, w)\) muss dominiert werden\\
Da \(N_3(v, w)\) keine Verbindung nach außen, kann man nur aus einer begrenzten lokalen Menge \(M\) auswählen.
\[
M := \{v, w\} \cup N_2(v, w) \cup N_3(v, w).
\]
\begin{lemma}
Sei \(G=(V,E)\) ein Graph und sei \(G'=(V',E')\) der Graph, der nach der Anwendung von Regel 2 auf G resultiert. Dann gilt \(\gamma(G)=\gamma(G')\). 
\end{lemma}
\end{frame}

\begin{frame}{Verschiedene Fälle analysieren}
\begin{block}{(1.1)}
\begin{itemize}
    \item Alle Knoten aus der Menge \(N_3(v, w)\) sind sowohl Nachbarn von \(v\) als auch von \(w\)
    \item Optimal: \(v\) oder \(w\) auswählen, um gesamte Menge zu dominieren, da für alle Knotenpaare \(x, y \in M\) gilt: \(N(x, y) \subseteq N(v, w)\)
    \item \textbf{Das Problem hier:} welche der beiden gewähl werden soll?
    \item \textbf{Lösung mit Gadget-Konstruktion:} Das Hinzufügen von zwei neue Knoten \(z, z'\) und die Kanten \(\{v,z\}, \{w,z\}, \{v,z'\}, \{w,z'\}\) zu G \(=>\) eine \textbf{Oder} modellieren
    \item Die Sicherheit vom Löschen von Knoten \(N_{3}(v,w)\) und \(N_{2}(v,w) \cap N(v) \cap N(w)\) ist gewährleistet, da die schon von \(v\) oder \(w\) dominiert
    \item Die Größe der dominierenden Menge ändert sich dabei nicht. 
\end{itemize}
\end{block}
\end{frame}

\begin{frame}{Verschiedene Fälle analysieren}
\begin{block}{(1.2)}
\begin{itemize}
    \item \(v\) dominiert die Menge \(N_3(v, w)\), aber \(w\) tut dies nicht 
    \item Optimal: \(v\) zu wählen,  da die Auswahl von \(v\) (möglicherweise zusammen mit \(w\)) mindestens so viele Knoten dominiert wie jede beliebige andere Kombination von zwei Knoten \(x, y\) aus der lokalen Menge \(M \setminus \left( \{v\} \cap \left( N_2(v, w) \cap N(w) \right) \right)
\)
    \item Das Hinzufügen von \(v'\) und die Kante \(\{v, v'\}\) stellt sicher, dass in der optimalen Lösung \(v\) enthalten sein muss
    \item \(w\) ist eventuell notwendig
    \item Wie im Fall (1.1): Das Löschen von \(N_{3}(v,w)\) und \(N_{2}(v,w) \cap N(v)\) aus G ist sicher, da die schon von \(v\) dominiert wurden
\end{itemize}
\end{block}

\begin{block}{(1.3)}
    Symmetrische Analyse von Fall (1.2)
\end{block}
\end{frame}

\begin{frame}{Verschiedene Fälle analysieren}
\begin{block}{(2)}
    \item Die Menge \(N_3(v, w)\) kann nicht von \(v\) oder \(w\) individuell dominiert werden  \(=>\) Hier werden 2 Knoten gebraucht
    \item Für alle Knotenpaare \(x, y \in M\) gilt: \(N(x, y) \subseteq N(v, w)\)
    \item Die Kanten \(\{v, v'\}\) und \(\{w, w'\}\) erzwingt die Aufnahme beider Knoten \(v\) und \(w\) in die dominierende Menge
    \item Das Löschen von \(N_3(v, w)\) und \(N_2(v, w)\) ist sicher, da alle Knoten bereits von \(v\) und \(w\) dominiert werden
    \item Die Größe von der dominierenden Menge ändert dabei nicht
\end{block}
\end{frame}

\begin{frame}[shrink]{Reduktionsregel 2: Laufzeit analysieren}
Die Anwendung von Regel 2 auf einen Graphen \(G = (V, E)\) mit \(n\) Knoten erfordert eine Laufzeit von \(O(n^2)\), falls G planar ist, und eine Laufzeit von \(O(n^4)\) im allgemeinen Fall. 
\begin{block}{Planare Graphen}
    \begin{itemize}
        \item Die Analyse der Nachbarschaftsbeziehungen innerhalb der kombinierten geschlossenen Nachbarschaft \(N[v, w] = N[v] \cup N[w]\)
        \item Die relevante Struktur ist der von \(N[v, w]\) induzierte Teilgraph \(G[N[v,w]]\) (auch planar)
        \item \textbf{Knotenanzahl:} \(k = |N[v, w]| \le |N[v]| + |N[w]| = (\deg(v)+1) + (\deg(w)+1)\) \(=>\) In \(O(\deg(v) + \deg(w))\) liegen
        \item \textbf{Kantenanzahl:} auch \(O(\deg(v) + \deg(w))\)
        \item Die Laufzeit für die Verarbeitung eines \textbf{einzelnen Paares} \((v, w)\) ebenfalls \(O(\deg(v) + \deg(w))\)
        \item Die Gesamtlaufzeit summiert die Kosten aller \(O(n^2)\) Knotenpaare:
        \[ \sum_{v,w \in V} O(\deg(v) + \deg(w)) = O\left(\sum_{v \in V} \left( \sum_{w \in V} \deg(v) + \sum_{w \in V} \deg(w) \right)\right) \]
        \item Die innere Summe \(\sum_{w \in V} \deg(w)\) ist nach dem Handschlaglemma \(2|E|\) \(=>\) liegt in \(O(n)\)
        \item Die Summe \(\sum_{w \in V} \deg(v)\) entspricht \(n \cdot \deg(v)\)
        \[ O\left(\sum_{v \in V} (n \cdot \deg(v) + O(n))\right) = O\left(n \sum_{v \in V} \deg(v) + n \cdot O(n)\right) = O(n \cdot O(n) + O(n^2)) = O(n^2) \]
        \item Somit liegt die Gesamtlaufzeit für planare Graphen in \(O(n^2)\)
    \end{itemize}
\end{block}
\end{frame}

\begin{frame}[shrink]{Reduktionsregel 2: Laufzeit analysieren}
\begin{block}{Allgemeine Graphen}
\begin{itemize}
    \item Die Anzahl der Knoten ist weiterhin \(k \in O(\deg(v) + \deg(w))\)
    \item Im schlimmsten Fall kann die Anzahl der Kanten in diesem Teilgraphen in der Größenordnung von \(O(k^2) = O((\deg(v) + \deg(w))^2)\) liegen
    \item Die Analyse für ein Paar \((v, w)\) erfordert eine Laufzeit von \(O((\deg(v) + \deg(w))^2)\)
    \item Gesamtlaufzeit:
    \[ \sum_{v,w \in V} O((\deg(v) + \deg(w))^2) \]
    \item Für jeden Knoten \(\deg(v) < n\) gilt \(=>\) \(O((n+n)^2) = O(n^2)\)
    \[ \sum_{v,w \in V} O(n^2) = O(n^2) \cdot O(n^2) = O(n^4) \]
    \item Daraus folgt, dass die Gesamtlaufzeit im allgemeinen Fall in \(O(n^4)\) liegt
\end{itemize}
\end{block}
\end{frame}

%---------------------------------------------------
\section{Eigenschaften + Komplexität}
%---------------------------------------------------
\begin{frame}[shrink]{Eigenschaften + Komplexität}
\begin{itemize}
  \item \textbf{Definition 10 – Reduzierter Graph:}  
    Ein Graph \( G = (V, E) \) ist \emph{reduziert}, wenn die Reduktionsregeln 1 und 2 nicht mehr anwendbar sind.
  
  \item \textbf{Strukturelle Eigenschaften reduzierter Graphen:}  
    \begin{itemize}
      \item Für jeden Knoten \( v \in V \) gilt \( N_3(v) = \emptyset \) (außer ggf. einem durch frühere Regelanwendung entstandenen Gadget-Knoten mit Grad 1).  
      \item Für jedes Knotenpaar \( (v, w) \) existiert ein Knoten in \( N_2(v, w) \cup N_3(v, w) \), der die gesamte Menge \( N_3(v, w) \) dominiert.  
      \item Andernfalls wäre Regel 2 noch anwendbar.
    \end{itemize}

  \item \textbf{Reduktionsprozess:}  
    \begin{itemize}
      \item Jede Regelanwendung entfernt Knoten, der Graph wird kleiner.  
      \item Maximal \( O(n) \) erfolgreiche Anwendungen (da endliche Knotenanzahl).  
      \item Der Prozess terminiert garantiert.
    \end{itemize}

  \item \textbf{Theorem 11 – Komplexität:}  
    \begin{itemize}
      \item Jeder Graph \( G \) kann in einen reduzierten Graphen \( G' \) mit gleicher Dominationszahl (\( \gamma(G) = \gamma(G') \)) transformiert werden.  
      \item \textbf{Laufzeit:}  
        \begin{itemize}
          \item \( O(n^3) \) für \emph{planare Graphen}  
          \item \( O(n^5) \) für \emph{allgemeine Graphen}
        \end{itemize}
    \end{itemize}
\end{itemize}
\end{frame}

%---------------------------------------------------
\section{Zusammenfassung}
%---------------------------------------------------

\begin{frame}{Zusammenfassung}
    \begin{block}{Zusammenfassung}
        \begin{itemize}
            \item Wir haben zwei einfache, auf lokalen Strukturen basierende Regeln zur Datenreduktion für Dominating Set kennengelernt. 
            \item Diese Regeln verkleinern die Problemgröße in polynomieller Zeit, ohne die Lösung zu verändern.
        \end{itemize}
    \end{block}
    
    \begin{exampleblock}{Praxisrelevanz und Alternativen}
        \begin{itemize}
            \item Die vorgestellten Regeln sind einfach umzusetzen und in der Praxis sehr effektiv. 
            \item Es gibt theoretisch schnellere Algorithmen (z.B. über Baumzerlegung) \cite{treedecompositions}, diese sind aber aufgrund hoher Konstanten und Komplexität oft weniger praxistauglich. 
        \end{itemize}
    \end{exampleblock}
\end{frame}

\begin{frame}[allowframebreaks]{References}
\bibliographystyle{plain}
\bibliography{reference}
\end{frame}

\begin{frame}
    \Huge
    \centering
    Vielen Dank für Ihre Aufmerksamkeit! \vspace{1cm}
    
    \Large
    Fragen?
\end{frame}




\end{document}
